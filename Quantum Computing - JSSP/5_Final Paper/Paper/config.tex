% +------------------------------------------------------------------------------+ %
% | EDIT THESE SETTINGS ACCORDING TO YOUR THESIS                                 | %
% +------------------------------------------------------------------------------+ %

\newcommand{\practical}{Quantum Computing Programmierung} % comment for master seminar
%\newcommand{\seminar}{Vertiefte Themen in Mobilen und Verteilten Systemen} % uncomment for master seminar

\newcommand{\authorA}{Ege Çimşir, Jonas Gottal, Sebastian Silva,}
\newcommand{\authorB}{Tobias Rohe, Viktoria Patapovich}
\newcommand{\supervisor}{Jonas Stein}

\newcommand{\thesistitle}{Projektarbeit am Lehrstuhl für mobile und verteilte Systeme}

\newcommand{\thesisabstract}{% \/ put your thesis abstract below \/
Das Job-Shop-Scheduling Problem ist NP-hart und beschäftigt gleichermaßen Wissenschaft und Industrie. Im Rahmen der QC-Optimization-Challenge hat die Firma TRUMPF eine Erweiterung dieser Problemstellung geboten: Hierbei geht es sowohl um die zeitliche Planung als auch Zuweisung verschiedener Bauteile auf eine Reihe von Maschinen. Zur Lösung dieses Problems stehen mehrere Quantencomputer -- darunter zwei Quantum Annealer und zwei Quantum Gate Model-Computer -- zur Verfügung. Die folgende Ausarbeitung zeigt, dass bisher lediglich hybride Annealing-Ansätze solche Problemstellungen bewältigen können.
}% <-- mind this closing brace!

\newcommand{\thesisauthorship}{% \/ describe who wrote what below \/
Ege Çimşir hat den Abschnitt~\ref{subsec:qm} verfasst. Den Abschnitt~\ref{sec:related} haben Tobias Rohe und Ege Çimşir gemeinsam verfasst. Den Abschnitt~\ref{subsec:gmodel} hat Tobias Rohe verfasst. Jonas Gottal und Viktoria Patapovich haben die Abschnitte~\ref{subsec:anneal},~\ref{subsec:pruning}, und~\ref{subsec:annealing} verfasst. Jonas Gottal und  Tobias Rohe haben den Abschnitt~\ref{subsec:approach} gemeinsam verfasst. Sebastian Silva hat die Abschnitte~\ref{subsec:jssp} und~\ref{subsec:qubo} verfasst. Den Abschnitt~\ref{subsec:qgm} haben Ege Çimşir, Sebastian Silva und Tobias Rohe gemeinsam verfasst. Die Abschnitte~\ref{sec:intro} und~\ref{sec:conclusion} wurden von allen Autoren gemeinsam verfasst.
Wir bedanken uns bei den Betreuern Jonas Stein, Sebastian Zielinski und Leo Sünkel sowie bei Frau Prof. Dr. Claudia Linnhoff-Popien für ein spannendes Semester und die Chance am \textit{QAR-Lab} mit Quantencomputern arbeiten zu dürfen. 
}% <-- mind this closing brace!


\selectlanguage{ngerman} %options: english, ngerman